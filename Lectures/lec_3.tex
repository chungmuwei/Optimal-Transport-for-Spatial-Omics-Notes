\chapter{Optimal Transport Problem Definition}
\lecture{3}{16 Sep. 17:00}{First note}
\section{Original Motivation of Optimal Transport}
The optimal transport problem was first described by the French mathematician Gaspard Monge in 1781. A worker with a shovel in hand has to move a large pile of sand lying on a construction site. The goal of the worker is to erect with all that sand a target pile with a prescribed shape (for example, that of a giant sand castle). Naturally, the worker wishes to minimize her total effort, quantified for instance as the total distance or time spent carrying shovelfuls of sand.\cite{COTFNT}

% The goal was to find the most efficient way to move earth to build military. Given a set of points, the current shape of the points, and the target shape. Assume that each point is atomic and current shape and target shape consist of same amount of points. What is the point mapping from current to target locations with the least total moving distance. The most efficient one-to-one mapping is called Monge map, and the total distance associated with it is called earth moving distance.
\section{Monge Problem}
\begin{definition}[Monge problem between discrete measures]\label{def}
  For discrete measure
  $$\alpha=\sum_{i=1}^n \mathbf{a}_i\delta_{x_i} \text{ and } \beta=\sum_{j=1}^m \mathbf{b}_j\delta_{y_j}$$,
  where $\mathbf{a}_i$ and $\mathbf{b}_j$ are weights, $x_i\in\mathcal{X}$ and $x_j\in\mathcal{Y}$ are locations, $\delta_x$ is a unit of mass which is infinitely concentrated at location $x$, called the Dirac at position $x$. The Monge problem seeks a map that associates each point $x_i$ with a single point $y_j$ and which must push the mass of $\alpha$ toward the mass of $\beta$, namely, such a map $T:\{x_1,\dots,x_n\}\rightarrow\{y_1,\dots,y_m\}$ must verity that 
  $$\forall j\in[m],\ b_j=\sum_{i:T(x_i)=y_j} \mathbf{a}_i$$,
  which is written as $T_\sharp\alpha=\beta$ in compact form. The objective of the map $T$ is to minimise some transportation cost, parameterised by a function $c(x,y)$ defined for points $(x,y)\in\mathcal{X}\cross\mathcal{Y}$, 
  $$\underset{T}{\text{min}}\Biggl\{\sum_i c(x_i,T(x_i))\ : \ T_\sharp\alpha=\beta\Biggr\}$$

  In the special case when $n=m$ (the number of points are equal in discrete measures) and all weights are uniform, that is $\mathbf{a}_i=\mathbf{b}_j=1/n$, the mass conservation constraints implies that $T$ is a \textbf{bijection} (\textbf{one-to-one matching}), and the Monge problem is equivalent to the optimal matching problem, where the cost matrix is 
  $$\mathbf{C}_{i,j}:=c(x_i,y_j)$$.
  When $n!=m$, Monge map may not exist between two discrete measures. This occurs when the target measure has more points than the source measure, $n<m$.
\end{definition}

\section{Kantorovich relaxation}
The monge problem is not always relevant to studying discrete measure in pratical problems. Formulated as a permutation problem, it can only be used to compare \textbf{uniform histograms} of the \textbf{same size}. Although it can be generalised to discrete measures with nonuniform weights using push-forward map, that formulation may be degenerate in the absence of feasible solutions satisfying the mass conservation constraint. 

The key idea of Kantorovich is to relax the deterministic nature of transportation, a source point $x_i$ can only be assigned to another point or location $T(x_i)$ only. Kantorovich proposed that the mass at any point $x_i$ can be potentially dispatched across several locations, moving from deterministic to probabilistic transport. It allows \textbf{mass splitting} from a source toward several targets. This partial matching is encode using a \textbf{coupling matrix} $\mathbf{P}\in\mathbb{R}_+^{n\cross m}$, where $\mathbf{P_{i,j}}$ describes the amount of mass flowing from bin $i$ toward bin $j$. Admissible couplings
$$
\mathbf{U}(\mathbf{a},\mathbf{b}):=\bigl\{ \mathbf{P}\in\mathbb{R}_+^{n\cross m}\ :\ \mathbf{P}\mathbbm{1}_m=\mathbf{a} \text{ and }\mathbf{P}^T\mathbbm{1}_n=\mathbf{b} \bigr\}
$$

The set of matrices $\mathbf{U}(\mathbf{a},\mathbf{b})$ is bounded and defined by $n+m$ constraints, and therefore is a convex polytope.


\begin{definition}[Kantorovich problem]\label{def:kantorovich}
$$
\mathrm{L_C(a,\ b):=\min_{P\in U(a,b)}\langle C,\ P \rangle := \sum_{i,j}C_{i,j}P_{i,j}}
$$
\end{definition}

\section{Dual Problem for the Kantorovich problem}

\begin{proposition}\label{prop}
  The Kantorovich problem (\autoref{def:kantorovich}) admits the dual
  \[
    \mathrm{L_C(a,b)=\max_{(f.g)\in R(C)}\langle f,\ a\rangle+\langle g,\ b\rangle}, 
  \]
  where $R(C)\in \mathbb{R}^{n\cross m}$
\end{proposition}

\section{Wasserstein distance}

\begin{definition}[Unbalanced mass between discrete measures]\label{def}
    
\end{definition}

- KL divergence, JS divergence, and others
- GAN architecture  and Wasserstein GAN
- Input convex neural networks
- Entropic regularisation
- Sinkhorn algorithm
- Low rank


% \begin{definition}[Natural number]\label{def}
%   We denote the set of \emph{natural numbers} as \(\mathbb{\MakeUppercase{n}} \).
% \end{definition}


% \begin{lemma}[Useful lemma]\label{lma}
%   Given the axioms of \hyperref[def]{natural numbers \(\mathbb{\MakeUppercase{n}} \)}, we have
%   \[
%     0\neq 1.
%   \]
% \end{lemma}
% \begin{proof}[An obvious proof]
%   Obvious.
% \end{proof}
% \begin{proposition}[Useful proposition]\label{prop}
%   From \autoref{lma}, we have
%   \[
%     0<1.
%   \]
% \end{proposition}
% \begin{exercise}
%   Prove that \(1 < 2\).
% \end{exercise}
% \begin{answer}
%   We note the following.
%   \begin{note}
%     We have \autoref{prop}! We can use it iteratively!
%   \end{note}
%   With the help of \autoref{lma}, this holds trivially.
% \end{answer}
% \begin{eg}
%   We now can have \(a < b\) for \(a < b\)!
% \end{eg}
% \begin{explanation}
%   Iteratively apply the exercise we did above.
% \end{explanation}
% \begin{remark}
%   We see that \autoref{prop} is really powerful. We now give an immediate application of it.
% \end{remark}

% \begin{theorem}[Mass-energy equivalence]\label{thm}
%   Given \autoref{prop}, we then have
%   \[
%     E = mc^2.
%   \]
% \end{theorem}
% \begin{proof}
%   The blank left for me is too small,\footnote{\url{https://en.wikipedia.org/wiki/Richard_Feynman}} hence we put the proof in \autoref{appendix}.
% \end{proof}

% From \autoref{thm}, we then have the following.
% \begin{corollary}[Riemann hypothesis]\label{col}
%   The real part of every nontrivial zero of the Riemann zeta function is \(\frac{1}{2}\), where the Riemann zeta function is just
%   \[
%     \zeta (s)=\sum _{n=1}^{\infty }{\frac {1}{n^{s}}}={\frac {1}{1^{s}}}+{\frac {1}{2^{s}}}+{\frac {1}{3^{s}}}+\cdots.
%   \]
% \end{corollary}
% \begin{proof}
%   The proof should be trivial, we left it to you.\todo{DIY}
% \end{proof}
% \begin{prev}
%   We see that \autoref{lma} is really helpful in the proof!
% \end{prev}

% \subsubsection{Internal Link}
% You should see all the common usages of internal links. Additionally, we can use citations as~\cite{newton1726philosophiae}, which just link to the reference page!

% \section{Figures}
% A simple demo for drawing:
% \begin{figure}[H]
%   \centering
%   \incfig{test}
%   \caption[Caption]{A \(3\)-torus.\footnotemark}
%   \label{fig:test}
% \end{figure}
% \footnotetext{For detailed information, please see \url{https://github.com/sleepymalc/VSCode-LaTeX-Inkscape}.}

% \section{Commutative Diagram}
% We can use the package \texttt{tikz-cd} to draw some commutative diagram.
% \begin{eg}
%   The cellular homology agrees with singular homology.
% \end{eg}
% \begin{explanation}
%   The following commutative diagram shows everything.

%   \adjustbox{scale=0.85,center}{%
%     \begin{tikzcd}[column sep=tiny]
%       &&&& {\color{red}0} \\
%       & {\color{red}0} && {\color{red}H_n(X^{n+1})\cong H_n(X)} \\
%       && {\color{red}H_n(X^n)} \\
%       \ldots & {\color{red}H_{n+1}(X^{n+1}, X^n)} && {\color{red}H_n(X^n, X^{n-1})} && {H_{n-1}(X^{n-1}, X^{n-2})} & \ldots \\
%       &&&& {\color{red}H_{n-1}(X^{n-1})} \\
%       &&& 0
%       \arrow[color={rgb,255:red,214;green,92;blue,92}, from=2-2, to=3-3]
%       \arrow["{\partial_{n+1}}", color={rgb,255:red,214;green,92;blue,92}, from=4-2, to=3-3]
%       \arrow[color={rgb,255:red,214;green,92;blue,92}, from=2-4, to=1-5]
%       \arrow[color={rgb,255:red,214;green,92;blue,92}, from=3-3, to=2-4]
%       \arrow[from=4-6, to=4-7]
%       \arrow["{d_n}"', from=4-4, to=4-6]
%       \arrow["{\partial_n}"', color={rgb,255:red,214;green,92;blue,92}, from=4-4, to=5-5]
%       \arrow["{j_{n-1}}"', from=5-5, to=4-6]
%       \arrow[from=6-4, to=5-5]
%       \arrow["{d_{n+1}}", from=4-2, to=4-4]
%       \arrow["{j_n}", color={rgb,255:red,214;green,92;blue,92}, from=3-3, to=4-4]
%       \arrow[from=4-1, to=4-2]
%     \end{tikzcd}
%   }
% \end{explanation}

% \section{Fancy Stuffs}
% With this header, you can achieve some cool things. For example, we can have multiple definitions under a parent environment, while maintains the numbering of definition. This is achieved by \texttt{definition*} environment with \texttt{definition} inside. For example, we can have the following.
% \begin{definition*}
%   We have the following number system.
%   \begin{definition}[Rational number]\label{def:rational}
%     The set of \emph{rational number}, denote as \(\mathbb{\MakeUppercase{q}} \).
%   \end{definition}
%   \begin{definition}[Real number]\label{def:real}
%     The set of \emph{real number}, denote as \(\mathbb{\MakeUppercase{r}} \).
%   \end{definition}
%   \begin{definition}[Complex number]\label{def:complex}
%     The set of \emph{complex number}, denote as \(\mathbb{\MakeUppercase{c}} \).
%   \end{definition}
% \end{definition*}

% \begin{note}
%   And indeed, we can still reference them correctly. For instance, we can use \hyperref[def:rational]{rational numbers} to define \hyperref[def:real]{real numbers} and then further use it to define \hyperref[def:complex]{complex numbers}.
% \end{note}

% Furthermore, we can completely control the name of our environments. We already saw we can name definition, lemma, proposition, corollary and theorem environment. In fact, we can also name remark, note, example and proof as follows.
% \begin{eg}[Interesting Example]\label{eg}
%   We note that \(1 \neq 2\)!
%   \begin{note}[Important note]
%     As a consequence, \(2 \neq 3\) also.
%   \end{note}

%   \begin{remark}[Easy observation]
%     We see that from here, we easily have the following theorem.
%     \begin{theorem}[Lebesgue Differentiation Theorem]\label{thm:Lebesgue-differentiation-theorem}
%       Let \(f\in L^1\), then
%       \[
%         \lim\limits_{r \to 0} \frac{1}{m(B(x, r))}\int_{B(x, r)}\left\vert f(y) - f(x) \right\vert   \,\mathrm{d}y = 0
%       \]
%       for a.e. \(x\).
%     \end{theorem}
%     \begin{proof}[An obvious proof of \autoref{thm:Lebesgue-differentiation-theorem}]
%       Obvious.
%     \end{proof}
%   \end{remark}
% \end{eg}
% As we can see, specifically for the \texttt{proof} environment, we allow \texttt{autoref} and \texttt{hyperref}. One can actually allow all example, note and remark environment's name to use reference, but I think that is overkilled. But this can be achieved by modify the header in an obvious way.\footnote{This time I mean it!}